% p03.tex --- CiP Problem Set Translation
% Based heavily off Wolfgang Mulzer's pset template. At some point I'm going to make a proper class/prelude out of this, 
% but I can't be bothered right now.

% Not sure I like the large font size - it's not a website -- but Mulzer seems to think it's a good choice
\documentclass[a4paper,11pt]{article}

% essential packages. I'm pretty sure most people have no idea what they need all of these for:
\usepackage{amsmath} % align environments etc.
\usepackage{amsfonts} % blackboard, fraktur etc.: everything textbook authors need to piss off their readers
\usepackage{xcolor} % we basically only want to use red, black and white, but oh well.
\usepackage{hyperref} % clickable links that look weird because I can't be bothered to style them
\usepackage{geometry} % to set margins etc.
\usepackage{fancyhdr} % for header/footer setup
\usepackage{titlesec} % to style headings
\usepackage{enumitem} % for list hierarchies
\usepackage{array} % table stuff
\usepackage{listings} % better than verbatim envs
\usepackage{multicol} % The normal title is way too big.
\usepackage{colortbl} % Alternating row colors


% Mulzer's list hierarchy:
\renewcommand{\labelenumi}{(\alph{enumi})}
% Bullet points for itemize, though
\setlist[itemize]{label=\textbullet, left=1em}

% Assorted pset vars
\newcommand{\course}{Concepts in Programming}
\newcommand{\pset}{3}
\newcommand{\due}{Friday, November 8, 2024, 12:00 PM}
% \newcommand{\authors}{Milan Andreew, Anton Platzek}

\title{
\hrule
\vspace{-0.5em}
\begin{multicols}{2}
\raggedright \large \textbf{\course}\\ % I'd have to mess around with the class file to actually make the title normal-sized, but this'll do
\large \textsc{Problem Set \pset} \\
\vspace{0.25em}

\raggedleft \small \textbf{Due:} \due \\
% \textbf{Authors:} \authors
\textit{Originals by W. Mulzer}
\end{multicols}
\vspace{-0.5em}
\hrule}
\date{}
\author{}

% Problems aren't actually environments, they're just `description' items.
\newcounter{pnr}
\setcounter{pnr}{1}
\newcommand{\problem}[2]{\vspace{0.3cm}\item[Problem \arabic{pnr}]\stepcounter{pnr} #1\hfill\emph{#2}}

% padding stuff - we want there to be less whitespace, but I haven't quite figured out how to solve the tutor 
% annotation issue - we could do boxes, but maybe dedicated space for margin notes is better? The current 
% equilibrium of "the tutor just dump PDF comments into free space" is suboptimal.
% Could also do our own footnote impl., but I'm not that good at TeX
\geometry{
	top=2cm,
	bottom=2cm,
	left=2.5cm,
	right=2.5cm
}

% basically the same as \pagestyle{empty}, except we have a custom footer (actually, we don't anymore, but nevermind)
\pagestyle{fancy}
\fancyhf{}
\renewcommand{\headrulewidth}{0pt}
\renewcommand{\footrulewidth}{0pt}
\fancyfoot[C]{\small \thepage}

% Header styling
\titleformat{\section}{\large\bfseries\uppercase}{\thesection}{1em}{}
\titleformat{\subsection}{\normalsize\bfseries\uppercase}{\thesubsection}{1em}{}

% Table styling
\renewcommand{\arraystretch}{1.3}
\newcolumntype{L}[1]{>{\raggedright\arraybackslash}p{#1}}
\newcolumntype{C}[1]{>{\centering\arraybackslash}p{#1}}
\newcolumntype{R}[1]{>{\raggedleft\arraybackslash}p{#1}}
\definecolor{rowgray}{gray}{0.9}

% Code styling - imo, code in psets is code written to be read, not run. Colorful syntax highlighting *may*
% be useful in an editor, but should be used sparingly in literature. Also, while most compilers can't deal with Unicode characters, humans can --- and we should be taking advantage of expressive notation as much as 
% possible. It's absolutely insane that there's no elegant way to make math mode monospaced inside listings, but I guess that one's TODO
\lstset{
	mathescape=true,
	backgroundcolor=\color[RGB]{245, 245, 245},
	frame=single,
	rulecolor=\color[RGB]{200, 200, 200},
	basicstyle=\ttfamily,
	columns=fullflexible,
	keepspaces=true,
% Still skeptical we really need this, but these'd be acceptable.
%	keywordstyle=\color[HTML]{007020},
%	commentstyle=\color[HTML]{60a0b0}\itshape,
%	stringstyle=\color[HTML]{4070a0},
	showstringspaces=false
	}


% Styling links --- for some reason underlines aren't working, so I've just made it thicker.
\hypersetup{
	colorlinks=false,
	linkbordercolor=black,
	urlbordercolor=black,
	pdfborderstyle={/S/U/W 2}
}

% \fancyfoot[R]{\small \textit{Original \TeX\ template courtesy of W. Mulzer}}

\begin{document}
\maketitle
\vspace{-2cm}
\begin{description}
\problem{Conditionals}{10 pts}

\begin{enumerate}
  \item Consider a programming language that provides only the primitive \texttt{if}-\texttt{then} construct (sans \texttt{elif} and \texttt{else} branches). Demonstrate how one might simulate a general \texttt{if}-\texttt{then}-\texttt{elif}-\texttt{else} construct.
  \item Research the \texttt{switch}-\texttt{case} construct in C and the \texttt{match}-\texttt{case} construct introduced in Python 3.10. Compare and contrast their properties.
  \item Describe how one might simulate C's \texttt{switch}-\texttt{case} construct using Python's conditional branching mechanisms.
\end{enumerate}

\problem{Loops}{10 pts}

\begin{enumerate}
\item Construct a program utilizing a \texttt{for} loop to generate the following numerical triangle:
\begin{lstlisting}
      1
     121
    12321
   1234321
  123454321
 12345654321
1234567654321
\end{lstlisting}
\item Implement a program that computes $\sum_{i=1}^n i$ utilizing a `for' loop. Additionally, consider if there exists a more elegant solution.
\item Research the semantics of Python's \texttt{break} and \texttt{continue} control primitives, providing illustrative examples. What are their behavioral characteristics in the context of nested loop structures?
\item Develop a program that continuously accepts numerical input until encountering a \href{https://en.wikipedia.org/wiki/Sentinel_value}{sentinel value} of 0, whereupon it shall compute and output the arithmetic mean of the provided sequence.
\end{enumerate}


\problem{A Random Walk}{10 pts}

Consider a stochastic process on $\mathbb{N}_0$ defined as follows:\\
Let $X_t$ be our position at time $t$, its initial state being $X_0=0$.

The transition probabilities are: 
\begin{align*}
&P(X_{t+1} | X_t = 0) = 1 \\
&P(X_{t+1} = z+1 | X_t = z) = P(X_{t+1} = z-1 | X_t = z) = \frac{1}{2}, \forall z > 0
\end{align*}

Implement this random walk in Python and use your program to answer the following questions. Consult the Python documentation to find a function for generating random numbers.

\begin{enumerate}
\item Calculate $\min{t: X_t = k}$ for $k \in \{10,20,30,50\}$. Come up with a hypothesis for a general formula with $k=n$
\item Study the distribution of $X_t$ for $t \in \{100, 100\}$. Try to plot it.
\end{enumerate}

\end{description}
\end{document}
