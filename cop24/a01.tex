\documentclass[a4paper,twoside,12pt]{article}
%\usepackage [reqno] {amsmath}
\usepackage{amsfonts,amstext}
\usepackage{amsmath}
\usepackage{babel}[english]
\usepackage{fullpage}
\usepackage{hyperref}

\newcommand{\ZETTELNUMMER}{1}
\newcommand{\ABGABEDATUM}{Friday, October 25, 2024, 12:00 PM}

\newcounter{AUFGNR}
\setcounter{AUFGNR}{1}
\newcommand{\AUFGABE}[2]{\vspace{0.3cm}\item[Problem \arabic{AUFGNR}]\stepcounter{AUFGNR} #1\hfill\emph{#2}}


\newcommand{\floor}[1]{\left\lfloor{#1}\right\rfloor}
\newcommand{\ceil}[1]{\left\lceil{#1}\right\rceil}
\newcommand{\half}[1]{\frac{#1}{2}}



\renewcommand{\labelenumi}{(\alph{enumi})}


\begin{document}
\pagestyle{empty}
\hrule\medskip
\rule{0ex}{0ex}\\[-1ex]
\ZETTELNUMMER. Problem Set

\smallskip
\noindent
\large
\textbf{Concepts of Programming}\hfill 
WiSe 2024/25 \\[0.5ex]
\normalsize
Wolfgang Mulzer

\medskip\hrule

\smallskip
\noindent
\textbf{Due} \ABGABEDATUM

\vskip 0.5cm

\begin{description}

		\AUFGABE{Algorithms}{10 Pts}
\begin{enumerate}

\item An algorithm is defined as a (1) \emph{finitely described},
(2) \emph{deterministic}, (3) \emph{effectively calculable} procedure
that transform an input into an output.

Elaborate on these three properties in the definition.
For each property, provide an example
where it is \emph{not} fulfilled.

\item Describe, in as much detail as possible, the steps to
  make a hard-boiled egg. Is your description an algorithm?
Why or why not?
\end{enumerate} 



\AUFGABE{A Game of Guessing Numbers}{10 Pts}

Werner and Hannelore play a game called \emph{Guess the Number}.
The rules are as follows:
\begin{enumerate}
\item Werner and Hannelore agree on a positive integer $n \geq 1$.
\item Werner thinks of a secret number $x\in \left[1,n\right]$. \footnote{The \emph{brackets} ($\left[ \right]$) denote a \href{https://en.wikipedia.org/wiki/Interval_(mathematics)}{\emph{closed interval}}, i.e. $1\leq x \leq n$.} 
\item Hannelore tries to guess $x$ by proposing numbers.
\item For each guess $y$, Werner responds with one of three statements:
  \begin {itemize}
  \item ``My number $x$ is greater than $y$.''
  \item ``My number $x$ is less than $y$.''
  \item ``Hit!'' (if $y=x$)
  \end {itemize}
\item The game ends when Hannelore guesses correctly.
\end{enumerate}

Assume Werner plays honestly.

Hannelore uses the following strategy:
\begin{itemize}
\item She maintains two variables, $a$ and $b$, intially set to $1$ and $n$, respectively.
\item In each round, Hannelore chooses a number $c\in\left[ a,b \right]$.
\item Based on Werner's response, Hannelore updates $a$ or $b$:
  \begin {itemize}
  \item If ``Hit!'', the game ends.
  \item If ``Less'', she sets $b = c-1$.
  \item If ``Greater'', she sets $a = c + 1$.
  \end {itemize}
\end{itemize}

\begin{enumerate}
  \item Provide a detailed example of this game for $n = 10$, showing
    Hannelore's strategy in action.
  \item Prove that Hannelore's strategy always succeeds in a finite number of rounds.

  \emph{Hint}: 
  Identify an invariant condition that holds at the beginning of each round.
  Your proof should address two key points:
  \begin{enumerate}
      \item[(i)] Hannelore makes ``progress'' in each round;
      \item[(ii)] Hannelore cannot ``miss'' Werner's number.
  \end{enumerate}
\end{enumerate}


\AUFGABE{First Steps in Python}{10 Pts}

\begin{enumerate}
\item Download and install Python (Version 3) on your computer.
\item Start the Python REPL and enter the following commands. 
  For each, explain what happens and provide a brief interpretation:
  \begin{enumerate}
    \item \begin{verbatim}a = 8 + 10\end{verbatim}
    \item \begin{verbatim}
help()
quit 
\end{verbatim}
    \item \begin{verbatim}3 + 5 * 7 == a - 2\end{verbatim}
    \item \begin{verbatim}
a = 40
3 + 5 * 7 == a - 2
\end{verbatim}
\item \begin{verbatim}print("KDP", str(a * 50 + 2*10 + 4 - 1) + ".\n")\end{verbatim}
    \item \begin{verbatim}True or (False and True)\end{verbatim}
    \item \begin{verbatim}
if a - 4 <= 5:
    print("Ja")
else:
    print("Nein")
          \end{verbatim}
    \item \begin{verbatim}
2 * (4 +

5)
\end{verbatim}
    \item \begin{verbatim}
for i in range(10):
    print(2 * i +  1)
    \end{verbatim}
    \item \begin{verbatim}exit()\end{verbatim}
  \end{enumerate}
\item
  Download the file \texttt{mystery.py} from the course website
  and import it into Python using the command
  \texttt{from mystery import mystery}.
  What happens when you enter:
  \begin{verbatim}
mystery("anna") 
mystery("banane")
mystery("caesar") 
\end{verbatim}

  Conduct further experiments and formulate a hypothesis about what the function ``mystery'' does.
\end{enumerate}


\end{description}
\end{document}
